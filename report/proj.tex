\documentclass{article} 

\usepackage{graphicx}
\usepackage{float}

\title{Projectile Position Prediction Report}

\begin{document}
\maketitle

\section{Assumption}
This problem is a free falling problem in physics. So actually we can just 
figure out the initial speed on x direction and y direction to do the 
simulation. If we are not told what scenario it is, we may still plot x and y 
versus t, then we will find that we can just fit a linear function for (t,x) 
and a quadratic function for y using lesat square error. So in order to make 
this problem more like machine learning problem than physic problem, we assume 
that we don't know this is a free falling problem, and the function relation 
cannot be gained from graph.

\section{Model and Evaluation}
According to our assumption, we cannot just fit the linear and quadratic model, 
but we still know there should be some functional relation between the 
projectile's current position and its two previous position, i.e. we can build 
a function x(i),y(i)=f(x(i-1),y(i-1),x(i-2),y(i-2)), where i is an arbitary 
time point. Due to neural network's high strength of approximating arbotary 
functions, we built a 4-layer neural network to predict positions. The input 
layer contains 4 units x(i-2),y(i-2),x(i-1),y(i-1) and the output layer 
contains the prediction x(i),y(i). In order to train the model, we split our 
data into training set and test set with 20\% of whole data in the test set. 
The evaluation function is the L2 error
\begin{equation}\label{eq1}
error=\frac{1}{2m}\sum_{i=1}^{m}((x_i-xpredict_i)^{2}+(y_i-ypredict_i)^{2})
\end{equation}
where m is the number of instances. After fine-tune parameters, we finally got 
a model with test error 0.312. We then use this model to continuely predict 
positions for projectile in A1. The predict result looks not good, that may be 
because we are lack of data in some areas.

\end{document}